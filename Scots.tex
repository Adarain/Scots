\documentclass{article}
\usepackage{color}
\usepackage{soul}
\definecolor{gray}{rgb}{0.84, 0.84, 0.84}
\newcommand{\hlv}[2][gray]{{\sethlcolor{#1}\hl{#2}}}
\usepackage{multirow}
\usepackage{fontspec}
\usepackage[hidelinks]{hyperref}
\usepackage{float}
\usepackage{multicol}
\usepackage{vowel}
\restylefloat{table}
\usepackage{libertine}
\usepackage{placeins}
\usepackage{graphicx}
\usepackage{lingmacros}
\graphicspath{ {images/} }
\newcommand{\smallc}[1]{{\addfontfeature{Letters=SmallCaps} #1}}

\author{Marcas Brian MacStiofáin Ó Mhaitiú Ó Domhnaill}
\title{\Huge{}Í Scóts Leíd\\\LARGE{}a swíf graímur}
\date{}
\begin{document}
\maketitle
\newpage
\tableofcontents
\listoftables
\newpage
\section{Pronunciation}
This document will use SSS spelling. For more information on the SSS orthography visit
\url{www.facebook.com/groups/SSSskreiv}
\subsection{Consonants}

\begin{table}[H]
\centering
\begin{tabular}{l|l|l|l}
    Scots&IPA&Example in Scots&Example in English\\\hline
    b&b&\hlv{b}reður&\hlv{b}rother\\
    c&k&\hlv{c}en&\hlv{c}at\\
    ch&x&lo\hlv{ch}&as in German Ba\hlv{ch}\\
    etc\dots&&&
\end{tabular}
\caption{Consonants}
\label{cons}
\end{table}

\subsection{Vowels}

\begin{table}[H]
\centering
\begin{tabular}{l|l|l|l}
    Scots&IPA&Example in Scots&Example in English\\\hline
    a&a&\hlv{a}turcap&\hlv{a}nt\\
    ai&e&st\hlv{ai}n&r\hlv{ai}n\\
    au&ɑ&\hlv{au}&t\hlv{al}k\\
    aȝ&ai&\hlv{aȝ}&\hlv{eye}\\
    etc\dots&&&
\end{tabular}
\caption{Vowels}
\label{vow}
\end{table}

Vowels in Scots have no inherent length, with the length being determined by the vowel's
environment. This is known as the Scottish Vowel Length Rule (SVLR) or Aitken's Law. The environment
for certain vowels being long are before voiced fricatives (such as v or ð), before r, before
morpheme boundaries and also in word-final open syllables.

\subsection{Stress and Pitch accent}

Although the workings of tone and pitch can vary a lot among different dialects, stress generally
tends to fall on the initial syllable. The Falkirk dialect has a pitch accent system where the initial
syllable takes on a  rising pitch, while the second syllable takes on a falling pitch. Any
determiners such as the articles \emph{i} "the" and \emph{a} "a/an", pronouns such as \emph{ma}
"my",
\emph{at} "that", \emph{is} "this" take on a falling pitch, as do any clitics and suffixes.

\end{document}
